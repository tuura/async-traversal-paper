\begin{abstract}

The availability of detailed protein-protein interaction networks or
``interactomes'' has made it possible to exploit network analysis techniques to
discover better drugs in faster and more efficient ways than ever before.
e-Therapeutics has developed a practical, in silico, approach to drug
discovery based on the construction and analysis of network representations of
disease mechanisms. Disease network construction and analysis is based on the
human interactome, a network of currently about 19K nodes linked by over 0.5M
interactions. Traversal operations on such a network are expensive and can
benefit from custom hardware acceleration. In this paper we outline two
approaches: a synchronous FPGA-based accelerator, which is very simple and
fast but limited to networks of thousands of proteins and 100s of thousands of
interactions, and an asynchronous alternative, which is more scalable and can
cope with networks comprising millions of nodes, but requires much more
sophisticated graph traversal algorithms. We present our current solution as a
challenge for the community: can you help us make it simpler?

\end{abstract}
